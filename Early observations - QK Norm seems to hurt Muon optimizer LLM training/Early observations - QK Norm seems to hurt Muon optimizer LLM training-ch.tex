\documentclass[11pt]{article}
\usepackage[UTF8]{ctex}
\usepackage{fontspec}
\usepackage{amsmath}
\usepackage{amssymb}
\usepackage{hyperref}
\usepackage[margin=1.2in]{geometry}
\usepackage{booktabs}
\usepackage{enumitem}
\usepackage{parskip}
\usepackage{setspace}
\usepackage{graphicx}
\usepackage{xcolor}
\usepackage{float}

\setstretch{1.15}
\setcounter{secnumdepth}{0}
\date{}

\title{QK-Norm 似乎损害了 Muon 优化器的大语言模型训练\thanks{作者感谢 Novita AI 为本研究提供计算资源。}}
\author{Vuk Rosić \\ \small \raisebox{-0.2\height}{\includegraphics[width=0.02\textwidth]{/root/llm-research-kit/.agent/skills/md-to-pdf/github.jpg}} \texttt{vukrosic}}

\begin{document}

\maketitle

今天我发现了一些奇怪的结果 —— QK-Norm + Muon 优化器虽然导致了更好的损失函数(loss)表现,但浪费了更多的计算资源(导致了更低秩的注意力头)。

\begin{quote}
\textbf{设置:} 88M 参数 LLM,22 层,64 维注意力头,Muon 优化器。
两次完全相同的实验 —— 一次在查询(queries)和键(keys)上使用了 QK-RMSNorm,另一次则没有。
\end{quote}

左图:QK-Norm 的损失函数看起来更好,但请看右侧面板,那是 \textbf{参与比 (Participation Ratio, PR)} —— 这是一个衡量每个注意力头 64 个可用维度中实际使用了多少维度的指标(其他维度塌缩 = 变成了其他维度的倍数,不携带新信息)。

\begin{figure}[H]
    \centering
    \includegraphics[width=0.8\textwidth]{comparison_20m.png}
    \caption*{Loss 与秩的对比}
\end{figure}

两者的 PR 都在下降。但使用了 QK-Norm 的模型塌缩得 \textbf{更快}。到 16M token 时,QK-Norm 模型的有效维度比不使用它的版本少了约 7\%。

这种差距还在不断扩大。以下是训练更久(25M token,独立实验)的情况:

\begin{figure}[H]
    \centering
    \includegraphics[width=0.8\textwidth]{../rank_study/pr_trajectory_4way.png}
    \caption*{25M Token 轨迹}
\end{figure}

目前来看,有效维度较少的模型损失函数依然略好。但它是否构建了一些脆弱的东西?

\section{塌缩发生在何处?}

这是最有趣的地方。塌缩在不同层之间并不是均匀的。

\textbf{Muon + QK-Norm (25M tokens) —— 某些层基本上已经“死”了:}

\begin{figure}[H]
    \centering
    \includegraphics[width=0.8\textwidth]{../rank_study/layer_rank_A1_Baseline_QK_Muon.png}
    \caption*{QK-Norm 逐层分解}
\end{figure}

看看第 1 层和第 3 层 —— 它们的 PR 降到了 \textbf{10} 以下。在 64 个可能的维度中,这些层只使用了大约 10 个。剩下的就是我们所谓的“幽灵计算” (ghost compute) —— GPU 正在对那些几乎没有任何贡献的维度进行数学运算。

\textbf{不带 QK-Norm 的 Muon (25M tokens) —— 表现均匀且具有活力:}

\begin{figure}[H]
    \centering
    \includegraphics[width=0.8\textwidth]{../rank_study/layer_rank_A2_NoQK_Muon.png}
    \caption*{无 QK-Norm 逐层分解}
\end{figure}

每一层的 PR 都保持在 45 以上。没有哪一层“死掉”了。整个网络的表征带宽非常均匀。

为什么归一化(normalization)会导致某些层塌缩而其他层不会?在 QK-Norm 的运行中,第 1 层和第 3 层有什么特殊之处?(尚未解决)

\section{悖论}

以下是我们测得的数据:

\begin{table}[H]
\centering
\begin{tabular}{@{}lll@{}}
\toprule
衡量指标 & QK-Norm & 无 QK-Norm \\ \midrule
验证集损失 (16M) & \checkmark \textbf{4.233} (更优) & 4.253 \\
有效秩 (16M) & 43.5 & \checkmark \textbf{46.8} (更高) \\
有效秩 (25M) & 30 (正在塌缩) & \checkmark \textbf{51} (稳定) \\
各层均匀性 & $\times$ 存在失效层 & \checkmark 所有层均活跃 \\ \bottomrule
\end{tabular}
\end{table}

结构上看起“更差”的模型实际上在预测 token 时表现略好。而内部表征更丰富的模型在损失函数上略微落后。

\subsection{待解决的问题}

\begin{itemize}
    \item \textbf{结构的优势最终会转化为更好的损失函数吗?}
    \item \textbf{QK-Norm 是否在“作弊”?} 它是否找到了一些低秩的快捷方式,虽然最小化了交叉熵,但牺牲了某些下游能力 —— 比如上下文学习(in-context learning)、推理能力或对分布式外(OOD)数据的泛化能力?
    \item \textbf{Muon 的正交化压力是否已经在做 QK-Norm 所做的事情了?}
    \item \textbf{PR 为 51 是否真的优于 PR 为 30?}
    \item \textbf{在 100M+ token 时会发生什么?}
\end{itemize}

可能发生的情况是 Muon 的正交化压力与 QK-Norm 的几何约束压力之间的博弈。

在没有归一化层的情况下,Muon 可以自由地推向高秩配置。也许模型起初训练稍慢,但其内部表征保持了多样性和分布性。

这种多样性是否真的对下游能力至关重要 —— 这是我们目前尚未回答的核心问题。

\end{document}
