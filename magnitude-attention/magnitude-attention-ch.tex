\documentclass[11pt]{ctexart}
\usepackage{fontspec}
\usepackage{amsmath, amssymb}
\usepackage{hyperref}
\usepackage{geometry}
\usepackage{booktabs}
\usepackage{enumitem}
\usepackage{parskip}
\usepackage{setspace}

\geometry{a4paper, margin=1.2in}
\setstretch{1.15}
\setcounter{secnumdepth}{0}

\title{幅度注意力:别让一组相似的键窃取你的概率质量}
\author{Vuk Rosić}
\date{\today}

\begin{document}

\maketitle

\section{注意力对几何视而不见}

标准注意力纯粹基于查询(Query)与单个键(Key)之间的成对兼容性来计算权重:

\[ \text{Attn}(Q,K,V) = \text{softmax}\!\left(\frac{QK^\top}{\sqrt{d_k}}\right)V \]

这种方式隐含地假设所有键提供的信息都是独立的。它对\textbf{键集合本身的内在几何结构}视而不见,尤其无法考虑\textbf{冗余性}。

在高维语义空间中,键往往聚集成密集的群组——重复的词元(Token)、近义词或反复出现的功能模式。我们将这些称为\textbf{分组键(Grouped Keys)}。

\subsection{劫持机制}
这些分组键利用了 softmax 归一化中的一个漏洞,即\textbf{概率劫持(Probability Hijacking)}。

\begin{enumerate}
    \item \textbf{形成群组}:系统性的冗余意味着许多键 $\{k_1, \dots, k_m\}$ 占据向量空间中的同一区域。因此,它们与查询的点积几乎相同:$q \cdot k_i \approx \text{常数}$。
    \item \textbf{劫持}:softmax 函数在分母中对这些分数的指数值求和。一个包含50个平庸键的簇——仅仅因为数量众多——就能积累巨大的概率质量,淹没一个高度相关的``独特''键。
\end{enumerate}

例如,考虑一个关键键 $k_{\text{unique}}$,其相关性分数为 $e^{q \cdot k_u} = 20$,以及一个包含50个冗余键的簇,每个键的相关性分数为 $e^{q \cdot k_i} = 1$。注意力权重变为:

\[ \text{Weight}(k_{\text{unique}}) = \frac{20}{20 + \underbrace{(1 + \dots + 1)}_{\text{50次}}} = \frac{20}{70} \approx \mathbf{0.28} \]

\[ \text{Weight}(\text{簇}) = \frac{50}{70} \approx \mathbf{0.71} \]

尽管独特键的相关性是任何单个簇成员的\textbf{20倍},但冗余群组仅凭数量优势就主导了注意力机制。

模型被迫过度关注信号的\textit{频率}而非其\textit{信息含量}。它浪费容量去检索同一冗余特征50次,同时淹没了仅出现一次的关键信号。

\section{数学解法:幅度(Magnitude)}

为了解决这个问题,我们需要一种方法来衡量一个集合中\textbf{``有效点数''}。

直觉上,如果你有3个键向量:
\begin{itemize}
    \item 如果它们彼此远离(正交),它们提供\textbf{3个信息单元}。
    \item 如果它们完全相同(克隆),它们仅提供\textbf{1个信息单元}。
    \item 如果它们有一定相似性,则提供\textbf{1到3之间}的信息单元。
\end{itemize}

能够捕捉这种连续``有效计数''的数学工具叫做\textbf{幅度(Magnitude)}。它为每个键分配一个权重 $\mu_j$,表示其独特贡献。

\subsection{工作原理:权重方程}

核心机制是一个线性系统,用于求解每个键的``独特性权重''。

\[ Z \boldsymbol{\mu} = \mathbf{1} \]

以下是每个分量的精确含义:

\begin{enumerate}
    \item \textbf{$Z$(相似度矩阵)}:这是一个 $N \times N$ 的矩阵,其中元素 $Z_{ij}$ 是一个介于0和1之间的数,衡量键 $i$ 和键 $j$ 之间的相似度。
    \begin{itemize}
        \item $Z_{ij} = 1$ 表示两个键完全相同。
        \item $Z_{ij} \approx 0$ 表示两个键完全不同(正交)。
    \end{itemize}

    \item \textbf{$\boldsymbol{\mu}$(未知权重)}:这是一个列向量,包含序列中\textbf{每个}键的权重:$\boldsymbol{\mu} = [\mu_1, \mu_2, \dots, \mu_N]^\top$。在求和公式中,我们用下标 $i$ 表示``当前键''(即我们正在检查其约束的键),用下标 $j$ 遍历其所有邻居。

    \item \textbf{$\mathbf{1}$(单位约束)}:这是一个全1向量,作为每一行的目标值。
\end{enumerate}

\subsubsection{矩阵系统的可视化}

为了理解 $i$ 和 $j$ 之间的关系,我们来看3个键($N=3$)的完整系统:

\[
\begin{bmatrix}
Z_{11} & Z_{12} & Z_{13} \\
Z_{21} & Z_{22} & Z_{23} \\
Z_{31} & Z_{32} & Z_{33}
\end{bmatrix}
\begin{bmatrix}
\mu_1 \\
\mu_2 \\
\mu_3
\end{bmatrix}
=
\begin{bmatrix}
1 \\
1 \\
1
\end{bmatrix}
\]

\begin{itemize}
    \item \textbf{第1行(i=1)}:$Z_{11}\mu_1 + Z_{12}\mu_2 + Z_{13}\mu_3 = 1$。第一个键($i=1$)必须平衡自身的权重 $\mu_1$ 与邻居2和3的加权相似度。
    \item \textbf{第2行(i=2)}:$Z_{21}\mu_1 + Z_{22}\mu_2 + Z_{23}\mu_3 = 1$。第二个键($i=2$)有自己的约束,对所有邻居 $j \in \{1, 2, 3\}$ 求和。
\end{itemize}

\textbf{方程的作用:}

该方程对集合中的每一个键 $i$ 施加严格的约束:

\[ \underbrace{\mu_i \cdot Z_{ii}}_{\text{自身贡献}} + \underbrace{\sum_{j \neq i} \mu_j \cdot Z_{ij}}_{\text{来自邻居的贡献}} = 1 \]

用通俗的话说:``我自身的权重加上所有邻居的加权相似度之和必须恰好等于1。''

这迫使产生一个权衡:
\begin{itemize}
    \item \textbf{如果一个键没有邻居($Z_{ij} \approx 0$):}(请记住,$Z_{ij}$ 是由距离计算得出的相似度,通常为 $e^{-\text{距离}^2}$,因此距离很远意味着 $Z \approx 0$)。第二项消失,方程简化为 $\mu_i \cdot 1 = 1$,因此 $\mu_i$ 必须为 \textbf{1}。该键保留完整权重。
    \item \textbf{如果一个键有许多相同的邻居($Z_{ij} \approx 1$):}第二项变得非常大,因为许多邻居为求和贡献了值。为了使总和保持等于1,权重 $\mu$ 必须严格下降。具体来说,如果有 $N$ 个相同的键,它们的权重必须降至 $1/N$,以使其总和保持等于1。
\end{itemize}

通过求解这个系统,我们可以精确地推导出每个键相对于整个群组的冗余程度。这些权重的总和 $|X| = \sum \mu_i$ 就是集合的\textbf{幅度(Magnitude)}。它告诉我们真正存在多少信息。

\subsection{相似度核的选择}

为了使这个系统在实践中可行,我们必须为相似度 $Z_{ij}$ 选择一个具体的公式。我们不能使用任意函数;我们需要一个能保证线性系统 $Z\boldsymbol{\mu}=\mathbf{1}$ 确实可解的函数。

我们使用\textbf{高斯核(Gaussian Kernel)}:

\[ Z_{ij} = e^{-t \cdot \|x_i - x_j\|^2} \]

这一特定选择至关重要,因为高斯核是\textbf{严格正定的(Strictly Positive Definite, SPD)}。

简单来说,这个性质确保相似度矩阵 $Z$ 始终是\textbf{可逆的}。如果没有这个性质,即使两个键只是略微相似,数学计算也可能``崩溃''(导致除以零错误或无穷多解)。高斯核保证只要你的词元不是完美的克隆,权重 $\boldsymbol{\mu}$ 就始终存在唯一且稳定的解。

(如果你想深入了解技术细节,可以研究:\textbf{``为什么高斯(RBF)核是严格正定的?''}、\textbf{``严格正定性如何保证矩阵可逆性?''}以及\textbf{``用严格正定矩阵求解线性系统的数值稳定性''}。)

有了这个稳定的基础,我们现在可以构建完整的注意力机制。

\subsection{关键性质}

\begin{table}[h]
\centering
\begin{tabular}{ll}
\toprule
\textbf{性质} & \textbf{描述} \\
\midrule
冗余抑制 & 密集簇中的点获得\textit{更低}的 $\mu_j$ \\
独特性放大 & 孤立/边界点获得\textit{更高}的 $\mu_j$ \\
信息可加性 & 总信息量是不相关部分的总和 \\
多样性上限 & $|X|$ 代表绝对最大多样性 \\
\bottomrule
\end{tabular}
\end{table}

幅度权重 $\mu_j$ 回答的问题是:\textbf{``在给定所有其他点的情况下,点 $j$ 贡献了多少独特的几何信息?''}

\section{幅度注意力:构建方法}

\subsection{第一步——键空间几何(高斯核)}

给定键 $\{k_1, \dots, k_n\}$,我们构建相似度矩阵 $Z$。与使用固定点积的标准注意力不同,我们使用带有\textbf{可学习尺度参数 $t$}的高斯核:

\[ Z_{jl} = \exp\!\left(-\,t \cdot \frac{\|k_j - k_l\|^2}{d_k}\right) \]

参数 $t$ 充当\textbf{几何分辨率控制器}。通过使 $t$ 可学习,模型可以自主调节其相似度度量的``锐度'':
\begin{itemize}
    \item \textbf{高 $t$ 值}会创建一个严格的过滤器,只有极其接近的向量才被视为冗余。
    \item \textbf{低 $t$ 值}会创建一个更宽泛的过滤器,允许模型抑制并非完全相同但语义相关的``近义词''或语义簇。
\end{itemize}

本质上,模型学习的是一个最优半径——在这个半径内,两条信息应被视为``相同'',从而将抑制机制的灵敏度调整到特定数据集的最佳状态。

\subsection{第二步——求解权重(迭代法)}

为了求解 $\boldsymbol{\mu}$,我们使用\textbf{截断共轭梯度法(Truncated Conjugate Gradient, CG)}来求解系统 $Z\boldsymbol{\mu} = \mathbf{1}$。

相比代价高昂的 $O(N^3)$ 矩阵求逆,共轭梯度法仅需3--5次迭代即可找到最优权重。这使得复杂度保持在 $O(N^2)$,与标准注意力的开销一致。

当键几乎完全相同时,系统可能在数值上变得不稳定。我们使用\textbf{吉洪诺夫正则化(Tikhonov Regularization)}($Z + \epsilon I$)来确保求解器始终能找到唯一且稳定的解,避免训练过程中梯度``爆炸''。

\subsection{第三步——幅度门控}

一旦我们获得了独特性权重 $\boldsymbol{\mu}$,我们在注意力求和之前对值($V$)施加一个\textbf{幅度门控(Magnitude Gate)}。

\[ \text{Gate}_j = \sigma(\beta \cdot \mu_j + \gamma) \]
\[ \text{Output} = \text{Attn}(Q, K, (\text{Gate} \cdot V)) \]

与仅调整注意力分数(只改变概率)的方法不同,这种方法\textbf{物理性地衰减}信号质量。通过在求和之前将每个值向量 $V_j$ 乘以门控值,冗余词元被缩放至接近零。它们实际上``缩小''或从隐藏状态中消失,防止其冗余特征在最终输出中累积并压过独特信息。

考虑一个包含50个重复词元的簇。它们的幅度权重将为 $\mu_j = 1/50 = 0.02$。如果幅度门控传递了这个权重,则整个簇的贡献变为:
\[ \sum_{50 \text{ 个词元}} \text{Score} \cdot (0.02 \cdot V) = 50 \cdot (\text{Score} \cdot 0.02 \cdot V) = \mathbf{1.0} \cdot \text{Score} \cdot V \]
50个词元的庞大数量在数学上被压缩,迫使模型将整个簇视为仅\textbf{1个信息单元}。

可学习的参数($\beta, \gamma$)允许模型决定冗余抑制的力度。模型学会只在几何独特性权重确实能提高性能的地方``信任''它们。

\end{document}
