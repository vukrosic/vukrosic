\documentclass[11pt]{article}
\usepackage{fontspec}
\usepackage{amsmath, amssymb}
\usepackage{hyperref}
\usepackage{geometry}
\usepackage{booktabs}
\usepackage{enumitem}
\usepackage{parskip}
\usepackage{setspace}

\geometry{a4paper, margin=1.2in}
\setstretch{1.15}
\setcounter{secnumdepth}{0}

\title{Magnitude Attention: Don't Let Group of Similar Keys Steal Your Probability Mass}
\author{Vuk Rosić}
\date{\today}

\begin{document}

\maketitle

\section{Attention is Blind to Geometry}

Standard attention calculates weights based purely on pairwise compatibility between the query and individual keys:

\[ \text{Attn}(Q,K,V) = \text{softmax}\!\left(\frac{QK^\top}{\sqrt{d_k}}\right)V \]

This formulation implicitly assumes that all keys provide independent information. It is blind to the \textbf{intrinsic geometry of the key set itself}. Specifically, it fails to account for \textbf{redundancy}.

In high-dimensional semantic spaces, keys often cluster into dense groups---repeated tokens, near-synonyms, or recurring functional patterns. We call these \textbf{grouped keys}.

\subsection{The Mechanism of Hijacking}
These grouped keys exploit a vulnerability in the softmax normalization known as \textbf{probability hijacking}.

\begin{enumerate}
    \item \textbf{Being Grouped}: Systematic redundancy means many keys $\{k_1, \dots, k_m\}$ occupy the same region in vector space. Consequently, they all produce the same or similar dot product with the query: $q \cdot k_i \approx \text{constant}$.
    \item \textbf{The Hijack}: The softmax function sums the exponentials of these scores in its denominator. A cluster of 50 mediocre keys---simply by virtue of being numerous---can accumulate a massive aggregate probability mass, swamping a single, highly relevant ``unique'' key.
\end{enumerate}

For example, consider a single critical key $k_{\text{unique}}$ with a high relevance score $e^{q \cdot k_u} = 20$, and a cluster of 50 redundant keys with low relevance scores $e^{q \cdot k_i} = 1$. The attention weights become:

\[ \text{Weight}(k_{\text{unique}}) = \frac{20}{20 + \underbrace{(1 + \dots + 1)}_{\text{50 times}}} = \frac{20}{70} \approx \mathbf{0.28} \]

\[ \text{Weight}(\text{Cluster}) = \frac{50}{70} \approx \mathbf{0.71} \]

Despite the unique key being \textbf{20$\times$ more relevant} than any individual cluster member, the redundant group dominates the attention mechanism by sheer volume.

The model is forced to over-attend to the \textit{frequency} of a signal rather than its \textit{information content}. It wastes capacity retrieving the same redundant feature 50 times, while drowning out the singular, critical signal that appears only once.

\section{The Mathematical Solution: Magnitude}

To fix this, we need a way to measure the \textbf{``Effective Number of Points''} in a set.

Intuitively, if you have 3 key vectors:
\begin{itemize}
    \item If they are all far apart (orthogonal), they provide \textbf{3 units of information}.
    \item If they are all identical (clones), they provide only \textbf{1 unit of information}.
    \item If they are somewhat similar, they provide somewhere between \textbf{1 and 3 units}.
\end{itemize}

The mathematical tool that captures this continuous ``effective count'' is called \textbf{Magnitude}. It assigns a weight $\mu_j$ to each key representing its unique contribution.

\subsection{How It Works: The Weighting Equation}

The core mechanism is a linear system that solves for the ``uniqueness weight'' of each key.

\[ Z \boldsymbol{\mu} = \mathbf{1} \]

Here is exactly what each component represents:

\begin{enumerate}
    \item \textbf{$Z$ (The Similarity Matrix)}: This is an $N \times N$ matrix where entry $Z_{ij}$ is a number between 0 and 1. It measures the similarity between key $i$ and key $j$.
    \begin{itemize}
        \item $Z_{ij} = 1$ means the keys are identical.
        \item $Z_{ij} \approx 0$ means the keys are completely different (orthogonal).
    \end{itemize}

    \item \textbf{$\boldsymbol{\mu}$ (The Unknown Weights)}: This is a single column vector containing one weight for \textbf{every} key in the sequence: $\boldsymbol{\mu} = [\mu_1, \mu_2, \dots, \mu_N]^\top$. In the summation formula, we use the index $i$ to denote the ``current key'' whose constraint we are checking, and the index $j$ to iterate over all its neighbors.

    \item \textbf{$\mathbf{1}$ (The Unit Constraint)}: This is a vector of all ones. It acts as the target for every row.
\end{enumerate}

\subsubsection{Visualizing the Matrix System}

To see the relationship between $i$ and $j$, let's look at the full system for 3 keys ($N=3$):

\[
\begin{bmatrix}
Z_{11} & Z_{12} & Z_{13} \\
Z_{21} & Z_{22} & Z_{23} \\
Z_{31} & Z_{32} & Z_{33}
\end{bmatrix}
\begin{bmatrix}
\mu_1 \\
\mu_2 \\
\mu_3
\end{bmatrix}
=
\begin{bmatrix}
1 \\
1 \\
1
\end{bmatrix}
\]

\begin{itemize}
    \item \textbf{Row 1 (i=1)}: $Z_{11}\mu_1 + Z_{12}\mu_2 + Z_{13}\mu_3 = 1$. The first key ($i=1$) must balance its own weight $\mu_1$ with the weighted similarity of neighbors 2 and 3.
    \item \textbf{Row 2 (i=2)}: $Z_{21}\mu_1 + Z_{22}\mu_2 + Z_{23}\mu_3 = 1$. The second key ($i=2$) has its own constraint, summing over all neighbors $j \in \{1, 2, 3\}$.
\end{itemize}

\textbf{What the Equation Does:}

The equation imposes a strict constraint on every single key $i$ in the set:

\[ \underbrace{\mu_i \cdot Z_{ii}}_{\text{Self-Contribution}} + \underbrace{\sum_{j \neq i} \mu_j \cdot Z_{ij}}_{\text{Contribution from Neighbors}} = 1 \]

In plain English: ``The sum of my own weight plus the weighted similarity of all my neighbors must exactly equal 1.''

This forces a trade-off:
\begin{itemize}
    \item \textbf{If a key has no neighbors ($Z_{ij} \approx 0$):} (Remember, $Z_{ij}$ is the similarity calculated from distance, typically $e^{-\text{distance}^2}$, so being far apart means $Z \approx 0$). The second term disappears. The equation simplifies to $\mu_i \cdot 1 = 1$, so $\mu_i$ must be \textbf{1}. The key retains full weight.
    \item \textbf{If a key has many identical neighbors ($Z_{ij} \approx 1$):} The second term becomes very large because there are many neighbors contributing to the sum. To keep the total equal to 1, the weights $\mu$ must essentially strictly decrease. Specifically, if there are $N$ identical keys, their weights must drop to $1/N$ so that their sum stays equal to 1.
\end{itemize}

By solving this system, we mathematically deduce exactly how redundant each key is relative to the entire group. The sum of these weights, $|X| = \sum \mu_i$, is the \textbf{Magnitude} of the set. It tells us truly how much information is present.

\subsection{The Choice of Similarity Kernel}

To make this system work in practice, we must choose a specific formula for the similarity $Z_{ij}$. We cannot just use any function; we need one that guarantees the linear system $Z\boldsymbol{\mu}=\mathbf{1}$ is actually solvable.

We use the \textbf{Gaussian Kernel}:

\[ Z_{ij} = e^{-t \cdot \|x_i - x_j\|^2} \]

This specific choice is critical because the Gaussian kernel is \textbf{Strictly Positive Definite (SPD)}.

In simple terms, this property ensures that the similarity matrix $Z$ is always \textbf{invertible}. Without it, if two keys were even slightly similar, the math could ``break'' (resulting in divide-by-zero errors or infinite solutions). The Gaussian kernel guarantees that as long as your tokens aren't perfect clones, there is always exactly one stable, unique answer for the weights $\boldsymbol{\mu}$.

(If you want to dive deeper into the technical mechanics, research: \textbf{``Why is the Gaussian (RBF) kernel strictly positive definite?''}, \textbf{``How does the SPD property guarantee matrix invertibility?''} and \textbf{``Numerical stability of solving linear systems with SPD matrices.''}).

With this stable foundation, we can now build the full attention mechanism.

\subsection{Key Properties}

\begin{table}[h]
\centering
\begin{tabular}{ll}
\toprule
\textbf{Property} & \textbf{Statement} \\
\midrule
Redundancy suppression & Points in dense clusters receive \textit{lower} $\mu_j$ \\
Uniqueness amplification & Isolated/boundary points receive \textit{higher} $\mu_j$ \\
Information Additivity & Total information is the sum of unrelated parts. \\
Diversity Limit & $|X|$ represents the absolute maximum variety. \\
\bottomrule
\end{tabular}
\end{table}

The magnitude weight $\mu_j$ answers: \textbf{``How much unique geometric information does point $j$ contribute, given all the other points?''}

\section{Magnitude Attention: The Construction}

\subsection{Step 1 --- Key-Space Geometry (Gaussian)}

Given keys $\{k_1, \dots, k_n\}$, we construct the similarity matrix $Z$. Unlike standard attention which uses a fixed dot product, we use a Gaussian kernel with a \textbf{learnable scale $t$}:

\[ Z_{jl} = \exp\!\left(-\,t \cdot \frac{\|k_j - k_l\|^2}{d_k}\right) \]

This parameter $t$ acts as a \textbf{geometrical resolution control}. By making $t$ learnable, the model can autonomously tune the ``sharpness'' of its similarity metric:
\begin{itemize}
    \item A \textbf{high $t$} creates a strict filter where only extremely close vectors are treated as redundant.
    \item A \textbf{low $t$} creates a broader filter, allowing the model to suppress ``near-synonyms'' or semantically related clusters that aren't exact copies.
\end{itemize}

Essentially, the model learns the optimal radius at which two pieces of information should be considered ``the same,'' tuning the sensitivity of the suppression mechanism to the specific dataset.

\subsection{Step 2 --- Solving for Weights (Iterative)}

To find $\boldsymbol{\mu}$, we solve the system $Z\boldsymbol{\mu} = \mathbf{1}$ using \textbf{Truncated Conjugate Gradient (CG)}.

Instead of a costly $O(N^3)$ matrix inversion, CG finds the optimal weights in just 3--5 iterations. This keeps the complexity at $O(N^2)$, matching the overhead of standard attention.

When keys are nearly identical, the system can become numerically unstable. We use \textbf{Tikhonov regularization} ($Z + \epsilon I$) to ensure the solver always finds a unique, stable solution without gradients ``exploding'' during training.

\subsection{Step 3 --- Magnitude Gating}

Once we have the uniqueness weights $\boldsymbol{\mu}$, we apply a \textbf{Magnitude Gate} to the values ($V$) before the attention sum.

\[ \text{Gate}_j = \sigma(\beta \cdot \mu_j + \gamma) \]
\[ \text{Output} = \text{Attn}(Q, K, (\text{Gate} \cdot V)) \]

Unlike methods that just shift attention scores (which only change probabilities), this \textbf{physically dampens} the signal mass. By multiplying each value vector $V_j$ by the gate before the sum, redundant tokens are scaled down toward zero. They effectively ``shrink'' or vanish from the hidden state, preventing their redundant features from accumulating and over-powering the unique information in the final output.

Consider a cluster of 50 duplicate tokens. Their magnitude weights will be $\mu_j = 1/50 = 0.02$. If the Magnitude Gate passes this weight through, the contribution of the entire cluster becomes:
\[ \sum_{50 \text{ tokens}} \text{Score} \cdot (0.02 \cdot V) = 50 \cdot (\text{Score} \cdot 0.02 \cdot V) = \mathbf{1.0} \cdot \text{Score} \cdot V \]
The sheer volume of the 50 tokens is mathematically collapsed, forcing the model to treat the entire cluster as just \textbf{1 unit of information}.

The learnable parameters ($\beta, \gamma$) allow the model to decide how aggressively to suppress redundancy. It learns to ``trust'' the geometric uniqueness weights only where they actually improve performance.

\end{document}
